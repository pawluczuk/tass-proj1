
% Default to the notebook output style

    


% Inherit from the specified cell style.




    
\documentclass{article}

    
    
    \usepackage{graphicx} % Used to insert images
    \usepackage{adjustbox} % Used to constrain images to a maximum size 
    \usepackage{color} % Allow colors to be defined
    \usepackage{enumerate} % Needed for markdown enumerations to work
    \usepackage{geometry} % Used to adjust the document margins
    \usepackage{amsmath} % Equations
    \usepackage{amssymb} % Equations
    \usepackage{eurosym} % defines \euro
    \usepackage[mathletters]{ucs} % Extended unicode (utf-8) support
    \usepackage[utf8x]{inputenc} % Allow utf-8 characters in the tex document
    \usepackage[T1]{fontenc}
    \usepackage[polish]{babel}
    \usepackage{fancyvrb} % verbatim replacement that allows latex
    \usepackage{grffile} % extends the file name processing of package graphics 
                         % to support a larger range 
    % The hyperref package gives us a pdf with properly built
    % internal navigation ('pdf bookmarks' for the table of contents,
    % internal cross-reference links, web links for URLs, etc.)
    \usepackage{hyperref}
    \usepackage{longtable} % longtable support required by pandoc >1.10
    \usepackage{booktabs}  % table support for pandoc > 1.12.2
    \usepackage{ulem} % ulem is needed to support strikethroughs (\sout)
    

    
    
    \definecolor{orange}{cmyk}{0,0.4,0.8,0.2}
    \definecolor{darkorange}{rgb}{.71,0.21,0.01}
    \definecolor{darkgreen}{rgb}{.12,.54,.11}
    \definecolor{myteal}{rgb}{.26, .44, .56}
    \definecolor{gray}{gray}{0.45}
    \definecolor{lightgray}{gray}{.95}
    \definecolor{mediumgray}{gray}{.8}
    \definecolor{inputbackground}{rgb}{.95, .95, .85}
    \definecolor{outputbackground}{rgb}{.95, .95, .95}
    \definecolor{traceback}{rgb}{1, .95, .95}
    % ansi colors
    \definecolor{red}{rgb}{.6,0,0}
    \definecolor{green}{rgb}{0,.65,0}
    \definecolor{brown}{rgb}{0.6,0.6,0}
    \definecolor{blue}{rgb}{0,.145,.698}
    \definecolor{purple}{rgb}{.698,.145,.698}
    \definecolor{cyan}{rgb}{0,.698,.698}
    \definecolor{lightgray}{gray}{0.5}
    
    % bright ansi colors
    \definecolor{darkgray}{gray}{0.25}
    \definecolor{lightred}{rgb}{1.0,0.39,0.28}
    \definecolor{lightgreen}{rgb}{0.48,0.99,0.0}
    \definecolor{lightblue}{rgb}{0.53,0.81,0.92}
    \definecolor{lightpurple}{rgb}{0.87,0.63,0.87}
    \definecolor{lightcyan}{rgb}{0.5,1.0,0.83}
    
    % commands and environments needed by pandoc snippets
    % extracted from the output of `pandoc -s`
    \providecommand{\tightlist}{%
      \setlength{\itemsep}{0pt}\setlength{\parskip}{0pt}}
    \DefineVerbatimEnvironment{Highlighting}{Verbatim}{commandchars=\\\{\}}
    % Add ',fontsize=\small' for more characters per line
    \newenvironment{Shaded}{}{}
    \newcommand{\KeywordTok}[1]{\textcolor[rgb]{0.00,0.44,0.13}{\textbf{{#1}}}}
    \newcommand{\DataTypeTok}[1]{\textcolor[rgb]{0.56,0.13,0.00}{{#1}}}
    \newcommand{\DecValTok}[1]{\textcolor[rgb]{0.25,0.63,0.44}{{#1}}}
    \newcommand{\BaseNTok}[1]{\textcolor[rgb]{0.25,0.63,0.44}{{#1}}}
    \newcommand{\FloatTok}[1]{\textcolor[rgb]{0.25,0.63,0.44}{{#1}}}
    \newcommand{\CharTok}[1]{\textcolor[rgb]{0.25,0.44,0.63}{{#1}}}
    \newcommand{\StringTok}[1]{\textcolor[rgb]{0.25,0.44,0.63}{{#1}}}
    \newcommand{\CommentTok}[1]{\textcolor[rgb]{0.38,0.63,0.69}{\textit{{#1}}}}
    \newcommand{\OtherTok}[1]{\textcolor[rgb]{0.00,0.44,0.13}{{#1}}}
    \newcommand{\AlertTok}[1]{\textcolor[rgb]{1.00,0.00,0.00}{\textbf{{#1}}}}
    \newcommand{\FunctionTok}[1]{\textcolor[rgb]{0.02,0.16,0.49}{{#1}}}
    \newcommand{\RegionMarkerTok}[1]{{#1}}
    \newcommand{\ErrorTok}[1]{\textcolor[rgb]{1.00,0.00,0.00}{\textbf{{#1}}}}
    \newcommand{\NormalTok}[1]{{#1}}
    
    % Additional commands for more recent versions of Pandoc
    \newcommand{\ConstantTok}[1]{\textcolor[rgb]{0.53,0.00,0.00}{{#1}}}
    \newcommand{\SpecialCharTok}[1]{\textcolor[rgb]{0.25,0.44,0.63}{{#1}}}
    \newcommand{\VerbatimStringTok}[1]{\textcolor[rgb]{0.25,0.44,0.63}{{#1}}}
    \newcommand{\SpecialStringTok}[1]{\textcolor[rgb]{0.73,0.40,0.53}{{#1}}}
    \newcommand{\ImportTok}[1]{{#1}}
    \newcommand{\DocumentationTok}[1]{\textcolor[rgb]{0.73,0.13,0.13}{\textit{{#1}}}}
    \newcommand{\AnnotationTok}[1]{\textcolor[rgb]{0.38,0.63,0.69}{\textbf{\textit{{#1}}}}}
    \newcommand{\CommentVarTok}[1]{\textcolor[rgb]{0.38,0.63,0.69}{\textbf{\textit{{#1}}}}}
    \newcommand{\VariableTok}[1]{\textcolor[rgb]{0.10,0.09,0.49}{{#1}}}
    \newcommand{\ControlFlowTok}[1]{\textcolor[rgb]{0.00,0.44,0.13}{\textbf{{#1}}}}
    \newcommand{\OperatorTok}[1]{\textcolor[rgb]{0.40,0.40,0.40}{{#1}}}
    \newcommand{\BuiltInTok}[1]{{#1}}
    \newcommand{\ExtensionTok}[1]{{#1}}
    \newcommand{\PreprocessorTok}[1]{\textcolor[rgb]{0.74,0.48,0.00}{{#1}}}
    \newcommand{\AttributeTok}[1]{\textcolor[rgb]{0.49,0.56,0.16}{{#1}}}
    \newcommand{\InformationTok}[1]{\textcolor[rgb]{0.38,0.63,0.69}{\textbf{\textit{{#1}}}}}
    \newcommand{\WarningTok}[1]{\textcolor[rgb]{0.38,0.63,0.69}{\textbf{\textit{{#1}}}}}
    
    
    % Define a nice break command that doesn't care if a line doesn't already
    % exist.
    \def\br{\hspace*{\fill} \\* }
    % Math Jax compatability definitions
    \def\gt{>}
    \def\lt{<}
    % Document parameters
    \title{Analiza statystyczna grafu przy użyciu standardowych narzędzi}
    \author{Monika Pawluczuk, nr albumu 246428}
    
    
    

    % Pygments definitions
    
\makeatletter
\def\PY@reset{\let\PY@it=\relax \let\PY@bf=\relax%
    \let\PY@ul=\relax \let\PY@tc=\relax%
    \let\PY@bc=\relax \let\PY@ff=\relax}
\def\PY@tok#1{\csname PY@tok@#1\endcsname}
\def\PY@toks#1+{\ifx\relax#1\empty\else%
    \PY@tok{#1}\expandafter\PY@toks\fi}
\def\PY@do#1{\PY@bc{\PY@tc{\PY@ul{%
    \PY@it{\PY@bf{\PY@ff{#1}}}}}}}
\def\PY#1#2{\PY@reset\PY@toks#1+\relax+\PY@do{#2}}

\expandafter\def\csname PY@tok@gd\endcsname{\def\PY@tc##1{\textcolor[rgb]{0.63,0.00,0.00}{##1}}}
\expandafter\def\csname PY@tok@gu\endcsname{\let\PY@bf=\textbf\def\PY@tc##1{\textcolor[rgb]{0.50,0.00,0.50}{##1}}}
\expandafter\def\csname PY@tok@gt\endcsname{\def\PY@tc##1{\textcolor[rgb]{0.00,0.27,0.87}{##1}}}
\expandafter\def\csname PY@tok@gs\endcsname{\let\PY@bf=\textbf}
\expandafter\def\csname PY@tok@gr\endcsname{\def\PY@tc##1{\textcolor[rgb]{1.00,0.00,0.00}{##1}}}
\expandafter\def\csname PY@tok@cm\endcsname{\let\PY@it=\textit\def\PY@tc##1{\textcolor[rgb]{0.25,0.50,0.50}{##1}}}
\expandafter\def\csname PY@tok@vg\endcsname{\def\PY@tc##1{\textcolor[rgb]{0.10,0.09,0.49}{##1}}}
\expandafter\def\csname PY@tok@m\endcsname{\def\PY@tc##1{\textcolor[rgb]{0.40,0.40,0.40}{##1}}}
\expandafter\def\csname PY@tok@mh\endcsname{\def\PY@tc##1{\textcolor[rgb]{0.40,0.40,0.40}{##1}}}
\expandafter\def\csname PY@tok@go\endcsname{\def\PY@tc##1{\textcolor[rgb]{0.53,0.53,0.53}{##1}}}
\expandafter\def\csname PY@tok@ge\endcsname{\let\PY@it=\textit}
\expandafter\def\csname PY@tok@vc\endcsname{\def\PY@tc##1{\textcolor[rgb]{0.10,0.09,0.49}{##1}}}
\expandafter\def\csname PY@tok@il\endcsname{\def\PY@tc##1{\textcolor[rgb]{0.40,0.40,0.40}{##1}}}
\expandafter\def\csname PY@tok@cs\endcsname{\let\PY@it=\textit\def\PY@tc##1{\textcolor[rgb]{0.25,0.50,0.50}{##1}}}
\expandafter\def\csname PY@tok@cp\endcsname{\def\PY@tc##1{\textcolor[rgb]{0.74,0.48,0.00}{##1}}}
\expandafter\def\csname PY@tok@gi\endcsname{\def\PY@tc##1{\textcolor[rgb]{0.00,0.63,0.00}{##1}}}
\expandafter\def\csname PY@tok@gh\endcsname{\let\PY@bf=\textbf\def\PY@tc##1{\textcolor[rgb]{0.00,0.00,0.50}{##1}}}
\expandafter\def\csname PY@tok@ni\endcsname{\let\PY@bf=\textbf\def\PY@tc##1{\textcolor[rgb]{0.60,0.60,0.60}{##1}}}
\expandafter\def\csname PY@tok@nl\endcsname{\def\PY@tc##1{\textcolor[rgb]{0.63,0.63,0.00}{##1}}}
\expandafter\def\csname PY@tok@nn\endcsname{\let\PY@bf=\textbf\def\PY@tc##1{\textcolor[rgb]{0.00,0.00,1.00}{##1}}}
\expandafter\def\csname PY@tok@no\endcsname{\def\PY@tc##1{\textcolor[rgb]{0.53,0.00,0.00}{##1}}}
\expandafter\def\csname PY@tok@na\endcsname{\def\PY@tc##1{\textcolor[rgb]{0.49,0.56,0.16}{##1}}}
\expandafter\def\csname PY@tok@nb\endcsname{\def\PY@tc##1{\textcolor[rgb]{0.00,0.50,0.00}{##1}}}
\expandafter\def\csname PY@tok@nc\endcsname{\let\PY@bf=\textbf\def\PY@tc##1{\textcolor[rgb]{0.00,0.00,1.00}{##1}}}
\expandafter\def\csname PY@tok@nd\endcsname{\def\PY@tc##1{\textcolor[rgb]{0.67,0.13,1.00}{##1}}}
\expandafter\def\csname PY@tok@ne\endcsname{\let\PY@bf=\textbf\def\PY@tc##1{\textcolor[rgb]{0.82,0.25,0.23}{##1}}}
\expandafter\def\csname PY@tok@nf\endcsname{\def\PY@tc##1{\textcolor[rgb]{0.00,0.00,1.00}{##1}}}
\expandafter\def\csname PY@tok@si\endcsname{\let\PY@bf=\textbf\def\PY@tc##1{\textcolor[rgb]{0.73,0.40,0.53}{##1}}}
\expandafter\def\csname PY@tok@s2\endcsname{\def\PY@tc##1{\textcolor[rgb]{0.73,0.13,0.13}{##1}}}
\expandafter\def\csname PY@tok@vi\endcsname{\def\PY@tc##1{\textcolor[rgb]{0.10,0.09,0.49}{##1}}}
\expandafter\def\csname PY@tok@nt\endcsname{\let\PY@bf=\textbf\def\PY@tc##1{\textcolor[rgb]{0.00,0.50,0.00}{##1}}}
\expandafter\def\csname PY@tok@nv\endcsname{\def\PY@tc##1{\textcolor[rgb]{0.10,0.09,0.49}{##1}}}
\expandafter\def\csname PY@tok@s1\endcsname{\def\PY@tc##1{\textcolor[rgb]{0.73,0.13,0.13}{##1}}}
\expandafter\def\csname PY@tok@kd\endcsname{\let\PY@bf=\textbf\def\PY@tc##1{\textcolor[rgb]{0.00,0.50,0.00}{##1}}}
\expandafter\def\csname PY@tok@sh\endcsname{\def\PY@tc##1{\textcolor[rgb]{0.73,0.13,0.13}{##1}}}
\expandafter\def\csname PY@tok@sc\endcsname{\def\PY@tc##1{\textcolor[rgb]{0.73,0.13,0.13}{##1}}}
\expandafter\def\csname PY@tok@sx\endcsname{\def\PY@tc##1{\textcolor[rgb]{0.00,0.50,0.00}{##1}}}
\expandafter\def\csname PY@tok@bp\endcsname{\def\PY@tc##1{\textcolor[rgb]{0.00,0.50,0.00}{##1}}}
\expandafter\def\csname PY@tok@c1\endcsname{\let\PY@it=\textit\def\PY@tc##1{\textcolor[rgb]{0.25,0.50,0.50}{##1}}}
\expandafter\def\csname PY@tok@kc\endcsname{\let\PY@bf=\textbf\def\PY@tc##1{\textcolor[rgb]{0.00,0.50,0.00}{##1}}}
\expandafter\def\csname PY@tok@c\endcsname{\let\PY@it=\textit\def\PY@tc##1{\textcolor[rgb]{0.25,0.50,0.50}{##1}}}
\expandafter\def\csname PY@tok@mf\endcsname{\def\PY@tc##1{\textcolor[rgb]{0.40,0.40,0.40}{##1}}}
\expandafter\def\csname PY@tok@err\endcsname{\def\PY@bc##1{\setlength{\fboxsep}{0pt}\fcolorbox[rgb]{1.00,0.00,0.00}{1,1,1}{\strut ##1}}}
\expandafter\def\csname PY@tok@mb\endcsname{\def\PY@tc##1{\textcolor[rgb]{0.40,0.40,0.40}{##1}}}
\expandafter\def\csname PY@tok@ss\endcsname{\def\PY@tc##1{\textcolor[rgb]{0.10,0.09,0.49}{##1}}}
\expandafter\def\csname PY@tok@sr\endcsname{\def\PY@tc##1{\textcolor[rgb]{0.73,0.40,0.53}{##1}}}
\expandafter\def\csname PY@tok@mo\endcsname{\def\PY@tc##1{\textcolor[rgb]{0.40,0.40,0.40}{##1}}}
\expandafter\def\csname PY@tok@kn\endcsname{\let\PY@bf=\textbf\def\PY@tc##1{\textcolor[rgb]{0.00,0.50,0.00}{##1}}}
\expandafter\def\csname PY@tok@mi\endcsname{\def\PY@tc##1{\textcolor[rgb]{0.40,0.40,0.40}{##1}}}
\expandafter\def\csname PY@tok@gp\endcsname{\let\PY@bf=\textbf\def\PY@tc##1{\textcolor[rgb]{0.00,0.00,0.50}{##1}}}
\expandafter\def\csname PY@tok@o\endcsname{\def\PY@tc##1{\textcolor[rgb]{0.40,0.40,0.40}{##1}}}
\expandafter\def\csname PY@tok@kr\endcsname{\let\PY@bf=\textbf\def\PY@tc##1{\textcolor[rgb]{0.00,0.50,0.00}{##1}}}
\expandafter\def\csname PY@tok@s\endcsname{\def\PY@tc##1{\textcolor[rgb]{0.73,0.13,0.13}{##1}}}
\expandafter\def\csname PY@tok@kp\endcsname{\def\PY@tc##1{\textcolor[rgb]{0.00,0.50,0.00}{##1}}}
\expandafter\def\csname PY@tok@w\endcsname{\def\PY@tc##1{\textcolor[rgb]{0.73,0.73,0.73}{##1}}}
\expandafter\def\csname PY@tok@kt\endcsname{\def\PY@tc##1{\textcolor[rgb]{0.69,0.00,0.25}{##1}}}
\expandafter\def\csname PY@tok@ow\endcsname{\let\PY@bf=\textbf\def\PY@tc##1{\textcolor[rgb]{0.67,0.13,1.00}{##1}}}
\expandafter\def\csname PY@tok@sb\endcsname{\def\PY@tc##1{\textcolor[rgb]{0.73,0.13,0.13}{##1}}}
\expandafter\def\csname PY@tok@k\endcsname{\let\PY@bf=\textbf\def\PY@tc##1{\textcolor[rgb]{0.00,0.50,0.00}{##1}}}
\expandafter\def\csname PY@tok@se\endcsname{\let\PY@bf=\textbf\def\PY@tc##1{\textcolor[rgb]{0.73,0.40,0.13}{##1}}}
\expandafter\def\csname PY@tok@sd\endcsname{\let\PY@it=\textit\def\PY@tc##1{\textcolor[rgb]{0.73,0.13,0.13}{##1}}}

\def\PYZbs{\char`\\}
\def\PYZus{\char`\_}
\def\PYZob{\char`\{}
\def\PYZcb{\char`\}}
\def\PYZca{\char`\^}
\def\PYZam{\char`\&}
\def\PYZlt{\char`\<}
\def\PYZgt{\char`\>}
\def\PYZsh{\char`\#}
\def\PYZpc{\char`\%}
\def\PYZdl{\char`\$}
\def\PYZhy{\char`\-}
\def\PYZsq{\char`\'}
\def\PYZdq{\char`\"}
\def\PYZti{\char`\~}
% for compatibility with earlier versions
\def\PYZat{@}
\def\PYZlb{[}
\def\PYZrb{]}
\makeatother


    % Exact colors from NB
    \definecolor{incolor}{rgb}{0.0, 0.0, 0.5}
    \definecolor{outcolor}{rgb}{0.545, 0.0, 0.0}



    
    % Prevent overflowing lines due to hard-to-break entities
    \sloppy 
    % Setup hyperref package
    \hypersetup{
      breaklinks=true,  % so long urls are correctly broken across lines
      colorlinks=true,
      urlcolor=blue,
      linkcolor=darkorange,
      citecolor=darkgreen,
      }
    % Slightly bigger margins than the latex defaults
    
    \geometry{verbose,tmargin=1in,bmargin=1in,lmargin=1in,rmargin=1in}
    

    \begin{document}
    
    \maketitle


Zbiór danych wybrany na podstawie nr albumu: \emph{Interakcje pomiędzy pracownikami małej firmy, zanotowane przez obserwatora.}

\section{Zadanie 1.}\label{zadanie-1.}

Wczytanie pobranego grafu. Graf został pobrany w formacie .pickle jako
obiekt igraph, a następnie wyeksportowany do formatu Pajek. Następnie
graf w formacie Pajek został wczytany jako obiekt networkx i na nim
zostały wykonane wszystkie obliczenia za pomocą pakietu networkx.

    \begin{Verbatim}[commandchars=\\\{\}]
{\color{incolor}In [{\color{incolor}1}]:} \PY{k+kn}{import} \PY{n+nn}{igraph}
        \PY{n}{A} \PY{o}{=} \PY{n}{igraph}\PY{o}{.}\PY{n}{Graph}\PY{o}{.}\PY{n}{Read\PYZus{}Pickle}\PY{p}{(}\PY{l+s}{\PYZdq{}}\PY{l+s}{bkoff.pickle}\PY{l+s}{\PYZdq{}}\PY{p}{)}
        \PY{n}{A}\PY{p}{[}\PY{l+s}{\PYZdq{}}\PY{l+s}{BKOFFB}\PY{l+s}{\PYZdq{}}\PY{p}{]}\PY{o}{.}\PY{n}{write\PYZus{}pajek}\PY{p}{(}\PY{l+s}{\PYZdq{}}\PY{l+s}{bkoffb.pajek}\PY{l+s}{\PYZdq{}}\PY{p}{)}
\end{Verbatim}

    \section{Zadanie 2.}\label{zadanie-2.}

Przekształcenie grafu. Po zaimportowaniu grafu do networkx, zostały z
niego usunięte zduplikowane krawędzie oraz graf został przekształcony na
nieskierowany i przy okazji wyeksportowany do formatu Pajek aby na nim
wykonywać zadania w programie Pajek.

Graf posiada 40 wierzchołków, które są połączone 238 krawędziami.

    \begin{Verbatim}[commandchars=\\\{\}]
{\color{incolor}In [{\color{incolor}2}]:} \PY{k+kn}{import} \PY{n+nn}{networkx} \PY{k+kn}{as} \PY{n+nn}{nx}
        \PY{n}{G} \PY{o}{=} \PY{n}{nx}\PY{o}{.}\PY{n}{read\PYZus{}pajek}\PY{p}{(}\PY{l+s}{\PYZdq{}}\PY{l+s}{bkoffb.pajek}\PY{l+s}{\PYZdq{}}\PY{p}{)}
        \PY{c}{\PYZsh{} przekształcenie na graf nieskierowany, bez zduplikowanych krawedzi}
        \PY{n}{G} \PY{o}{=} \PY{n}{nx}\PY{o}{.}\PY{n}{Graph}\PY{p}{(}\PY{n}{G}\PY{p}{)}
        \PY{n}{nx}\PY{o}{.}\PY{n}{write\PYZus{}pajek}\PY{p}{(}\PY{n}{G}\PY{p}{,} \PY{l+s}{\PYZdq{}}\PY{l+s}{bkoffb\PYZhy{}networkx.pajek}\PY{l+s}{\PYZdq{}}\PY{p}{)}
        \PY{n}{G} \PY{o}{=} \PY{n}{nx}\PY{o}{.}\PY{n}{read\PYZus{}pajek}\PY{p}{(}\PY{l+s}{\PYZdq{}}\PY{l+s}{bkoffb\PYZhy{}networkx.pajek}\PY{l+s}{\PYZdq{}}\PY{p}{)}
        \PY{k}{print}\PY{p}{(}\PY{l+s}{\PYZdq{}}\PY{l+s}{Rozmiar grafu: }\PY{l+s}{\PYZdq{}}\PY{p}{,} \PY{n+nb}{len}\PY{p}{(}\PY{n}{G}\PY{o}{.}\PY{n}{edges}\PY{p}{(}\PY{p}{)}\PY{p}{)}\PY{p}{,} \PY{l+s}{\PYZdq{}}\PY{l+s}{Rzad grafu: }\PY{l+s}{\PYZdq{}}\PY{p}{,} \PY{n+nb}{len}\PY{p}{(}\PY{n}{G}\PY{o}{.}\PY{n}{nodes}\PY{p}{(}\PY{p}{)}\PY{p}{)}\PY{p}{)}
\end{Verbatim}

    \begin{Verbatim}[commandchars=\\\{\}]
('Rozmiar grafu: ', 238, 'Rzad grafu: ', 40)
    \end{Verbatim}

    \section{Zadanie 3.}\label{zadanie-3.}

\subsection{Networkx}\label{networkx}

Wyznacz składowe spójne. Ile ich jest; jaki jest rząd i rozmiar
największej z nich?

Istnieje tylko jedna składowa spójna grafu, a więc automatycznie jest
ona największa. Zawiera ona w sobie wszystkie wierzchołki, więc jej rząd
wynosi 40. Podobnie z rozmiarem, który wynosi 238 (wszystkie istniejące
krawędzie).

Liczba składowych spójnych:

    \begin{Verbatim}[commandchars=\\\{\}]
{\color{incolor}In [{\color{incolor}3}]:} \PY{n}{nx}\PY{o}{.}\PY{n}{number\PYZus{}connected\PYZus{}components}\PY{p}{(}\PY{n}{G}\PY{p}{)}
\end{Verbatim}

            \begin{Verbatim}[commandchars=\\\{\}]
{\color{outcolor}Out[{\color{outcolor}3}]:} 1
\end{Verbatim}
        
    Posortowana wg. długości tablica wszystkich składowych spójnych:

    \begin{Verbatim}[commandchars=\\\{\}]
{\color{incolor}In [{\color{incolor}4}]:} \PY{p}{[}\PY{n+nb}{len}\PY{p}{(}\PY{n}{c}\PY{p}{)} \PY{k}{for} \PY{n}{c} \PY{o+ow}{in} \PY{n+nb}{sorted}\PY{p}{(}\PY{n}{nx}\PY{o}{.}\PY{n}{connected\PYZus{}components}\PY{p}{(}\PY{n}{G}\PY{p}{)}\PY{p}{,} \PY{n}{key}\PY{o}{=}\PY{n+nb}{len}\PY{p}{,} \PY{n}{reverse}\PY{o}{=}\PY{n+nb+bp}{True}\PY{p}{)}\PY{p}{]}
\end{Verbatim}

            \begin{Verbatim}[commandchars=\\\{\}]
{\color{outcolor}Out[{\color{outcolor}4}]:} [40]
\end{Verbatim}
        
    Lista grafów zawierająca składowe spójne jako oddzielne podgrafy i
liczba krawędzi pierwszej (i jedynej) z nich:

    \begin{Verbatim}[commandchars=\\\{\}]
{\color{incolor}In [{\color{incolor}5}]:} \PY{n}{graphs} \PY{o}{=} \PY{n+nb}{list}\PY{p}{(}\PY{n}{nx}\PY{o}{.}\PY{n}{connected\PYZus{}component\PYZus{}subgraphs}\PY{p}{(}\PY{n}{G}\PY{p}{)}\PY{p}{)}
        \PY{n}{graphs}\PY{p}{[}\PY{l+m+mi}{0}\PY{p}{]}\PY{o}{.}\PY{n}{number\PYZus{}of\PYZus{}edges}\PY{p}{(}\PY{p}{)}
\end{Verbatim}

            \begin{Verbatim}[commandchars=\\\{\}]
{\color{outcolor}Out[{\color{outcolor}5}]:} 238
\end{Verbatim}
        
    \subsection{Pajek}\label{pajek}

Po wczytaniu grafu, używając polecenia: Network -\textgreater{} Create
Partition -\textgreater{} Components -\textgreater{} Strong został
utworzony jeden podgraf, zawierający wszystkie 40 wierzchołków. Wynik
jest więc zgodny z wynikami otrzymanymi w NetworkX.

Wydruk Pajeka:

    \begin{Verbatim}[commandchars=\\\{\}]
{\color{incolor}In [{\color{incolor} }]:} \PY{o}{==}\PY{o}{==}\PY{o}{==}\PY{o}{==}\PY{o}{==}\PY{o}{==}\PY{o}{==}\PY{o}{==}\PY{o}{==}\PY{o}{==}\PY{o}{==}\PY{o}{==}\PY{o}{==}\PY{o}{==}\PY{o}{==}\PY{o}{==}\PY{o}{==}\PY{o}{==}\PY{o}{==}\PY{o}{==}\PY{o}{==}\PY{o}{==}\PY{o}{==}\PY{o}{==}\PY{o}{==}\PY{o}{==}\PY{o}{==}\PY{o}{==}\PY{o}{==}\PY{o}{==}\PY{o}{==}\PY{o}{==}\PY{o}{==}\PY{o}{==}\PY{o}{==}\PY{o}{==}\PY{o}{==}\PY{o}{==}\PY{o}{==}
        \PY{n}{Strong} \PY{n}{Components}
        \PY{o}{==}\PY{o}{==}\PY{o}{==}\PY{o}{==}\PY{o}{==}\PY{o}{==}\PY{o}{==}\PY{o}{==}\PY{o}{==}\PY{o}{==}\PY{o}{==}\PY{o}{==}\PY{o}{==}\PY{o}{==}\PY{o}{==}\PY{o}{==}\PY{o}{==}\PY{o}{==}\PY{o}{==}\PY{o}{==}\PY{o}{==}\PY{o}{==}\PY{o}{==}\PY{o}{==}\PY{o}{==}\PY{o}{==}\PY{o}{==}\PY{o}{==}\PY{o}{==}\PY{o}{==}\PY{o}{==}\PY{o}{==}\PY{o}{==}\PY{o}{==}\PY{o}{==}\PY{o}{==}\PY{o}{==}\PY{o}{==}\PY{o}{==}
         \PY{n}{Working}\PY{o}{.}\PY{o}{.}\PY{o}{.}
         \PY{n}{Number} \PY{n}{of} \PY{n}{components}\PY{p}{:} \PY{l+m+mi}{1}
         \PY{n}{Size} \PY{n}{of} \PY{n}{the} \PY{n}{largest} \PY{n}{component}\PY{p}{:} \PY{l+m+mi}{40} \PY{n}{vertices} \PY{p}{(}\PY{l+m+mf}{100.000}\PY{o}{\PYZpc{}}\PY{p}{)}\PY{o}{.}
         \PY{n}{Time} \PY{n}{spent}\PY{p}{:}  \PY{l+m+mi}{0}\PY{p}{:}\PY{l+m+mo}{00}\PY{p}{:}\PY{l+m+mo}{00}
\end{Verbatim}

    \section{Zadanie 4.}\label{zadanie-4.}

Wykreśl graf w Pajeku.

Wielkość wierzchołka w grafie jest proporcjonalna do stopnia
centralności w grafie. Zostały zmienione kolory krawędzi i wierzchołków
w celu lepszej czytelności grafu.

Został użyty rozkład wierzchołków typu Energy, Kamada-Kawai, Separate
Components. 

\adjustimage{max size={0.9\linewidth}{0.9\paperheight}}{not_files/pajek-network.jpg}

    \section{Zadanie 5.}\label{zadanie-5.}

Znajdź pięć wierzchołków o największych wartościach parametrów.

    \subsection{\texorpdfstring{Bliskości (\emph{closeness
centrality})}{bliskości (closeness centrality)}}\label{rangi-closeness-centrality}

Wierzchołki z numerami: 1, 2, 12, 13, 15.

    \begin{Verbatim}[commandchars=\\\{\}]
{\color{incolor}In [{\color{incolor}7}]:} \PY{n}{cent\PYZus{}dict} \PY{o}{=} \PY{n}{nx}\PY{o}{.}\PY{n}{closeness\PYZus{}centrality}\PY{p}{(}\PY{n}{G}\PY{p}{)}
        \PY{n}{cent\PYZus{}items}\PY{o}{=}\PY{p}{[}\PY{p}{(}\PY{n}{b}\PY{p}{,}\PY{n}{a}\PY{p}{)} \PY{k}{for} \PY{p}{(}\PY{n}{a}\PY{p}{,}\PY{n}{b}\PY{p}{)} \PY{o+ow}{in} \PY{n}{cent\PYZus{}dict}\PY{o}{.}\PY{n}{iteritems}\PY{p}{(}\PY{p}{)}\PY{p}{]}
        \PY{n}{cent\PYZus{}items}\PY{o}{.}\PY{n}{sort}\PY{p}{(}\PY{n}{reverse}\PY{o}{=}\PY{n+nb+bp}{True}\PY{p}{)}
        \PY{n}{cent\PYZus{}items}\PY{p}{[}\PY{l+m+mi}{0}\PY{p}{:}\PY{l+m+mi}{5}\PY{p}{]}
\end{Verbatim}

            \begin{Verbatim}[commandchars=\\\{\}]
{\color{outcolor}Out[{\color{outcolor}7}]:} [(0.6724137931034483, u'n13'),
         (0.6610169491525424, u'n2'),
         (0.65, u'n15'),
         (0.65, u'n12'),
         (0.65, u'n1')]
\end{Verbatim}
        
    \subsection{\texorpdfstring{Pośrednictwa (\emph{betweenness
centrality})}{pośrednictwa (betweenness centrality)}}\label{poux15brednictwa-betweenness-centrality}

Wierzchołki z numerami: 1, 2, 12, 13, 21.

    \begin{Verbatim}[commandchars=\\\{\}]
{\color{incolor}In [{\color{incolor}8}]:} \PY{n}{betweenness\PYZus{}dict} \PY{o}{=} \PY{n}{nx}\PY{o}{.}\PY{n}{betweenness\PYZus{}centrality}\PY{p}{(}\PY{n}{G}\PY{p}{)}
        \PY{n}{betweenness\PYZus{}items} \PY{o}{=} \PY{p}{[}\PY{p}{(}\PY{n}{b}\PY{p}{,}\PY{n}{a}\PY{p}{)} \PY{k}{for} \PY{p}{(}\PY{n}{a}\PY{p}{,}\PY{n}{b}\PY{p}{)} \PY{o+ow}{in} \PY{n}{betweenness\PYZus{}dict}\PY{o}{.}\PY{n}{iteritems}\PY{p}{(}\PY{p}{)}\PY{p}{]}
        \PY{n}{betweenness\PYZus{}items}\PY{o}{.}\PY{n}{sort}\PY{p}{(}\PY{n}{reverse}\PY{o}{=}\PY{n+nb+bp}{True}\PY{p}{)}
        \PY{n}{betweenness\PYZus{}items}\PY{p}{[}\PY{l+m+mi}{0}\PY{p}{:}\PY{l+m+mi}{5}\PY{p}{]}
\end{Verbatim}

            \begin{Verbatim}[commandchars=\\\{\}]
{\color{outcolor}Out[{\color{outcolor}8}]:} [(0.06225133611416058, u'n1'),
         (0.04532200555595849, u'n12'),
         (0.043181681821834246, u'n2'),
         (0.041438373178073105, u'n13'),
         (0.039945056663322924, u'n21')]
\end{Verbatim}
        
    \subsection{\texorpdfstring{Rangi (\emph{degree
centrality})}{rangi (degree centrality)}}\label{rangi-degree-centrality}

Wierzchołki z numerami: 1, 2, 12, 13, 15.

    \begin{Verbatim}[commandchars=\\\{\}]
{\color{incolor}In [{\color{incolor}9}]:} \PY{n}{degree\PYZus{}dict} \PY{o}{=} \PY{n}{nx}\PY{o}{.}\PY{n}{degree\PYZus{}centrality}\PY{p}{(}\PY{n}{G}\PY{p}{)}
        \PY{n}{degree\PYZus{}items} \PY{o}{=} \PY{p}{[}\PY{p}{(}\PY{n}{b}\PY{p}{,}\PY{n}{a}\PY{p}{)} \PY{k}{for} \PY{p}{(}\PY{n}{a}\PY{p}{,}\PY{n}{b}\PY{p}{)} \PY{o+ow}{in} \PY{n}{degree\PYZus{}dict}\PY{o}{.}\PY{n}{iteritems}\PY{p}{(}\PY{p}{)}\PY{p}{]}
        \PY{n}{degree\PYZus{}items}\PY{o}{.}\PY{n}{sort}\PY{p}{(}\PY{n}{reverse}\PY{o}{=}\PY{n+nb+bp}{True}\PY{p}{)}
        \PY{n}{degree\PYZus{}items}\PY{p}{[}\PY{l+m+mi}{0}\PY{p}{:}\PY{l+m+mi}{5}\PY{p}{]}
\end{Verbatim}

            \begin{Verbatim}[commandchars=\\\{\}]
{\color{outcolor}Out[{\color{outcolor}9}]:} [(0.5128205128205128, u'n13'),
         (0.48717948717948717, u'n2'),
         (0.48717948717948717, u'n12'),
         (0.4615384615384615, u'n15'),
         (0.4615384615384615, u'n1')]
\end{Verbatim}
        
    \section{Zadanie 6.}\label{zadanie-6.}

Znajdź wszystkie największe kliki - ile ich jest i jakiego rzędu?

W tym grafie istnieją 142 największe kliki, o rzędach: 

\begin{itemize}
  \item 2 (4 kliki o rzędzie równym dwa)
  \item 3 (41)
  \item 4 (60)
  \item 5 (26)
  \item 6 (11).
\end{itemize}

Ilość największych klik:

    \begin{Verbatim}[commandchars=\\\{\}]
{\color{incolor}In [{\color{incolor}9}]:} \PY{c}{\PYZsh{} Returns the number of maximal cliques in G.}
        \PY{n}{nx}\PY{o}{.}\PY{n}{graph\PYZus{}number\PYZus{}of\PYZus{}cliques}\PY{p}{(}\PY{n}{G}\PY{p}{)}
\end{Verbatim}

            \begin{Verbatim}[commandchars=\\\{\}]
{\color{outcolor}Out[{\color{outcolor}9}]:} 142
\end{Verbatim}
        
    Rzędy największych klik:

    \begin{Verbatim}[commandchars=\\\{\}]
{\color{incolor}In [{\color{incolor}10}]:} \PY{n}{max\PYZus{}cliques} \PY{o}{=} \PY{n+nb}{list}\PY{p}{(}\PY{n}{nx}\PY{o}{.}\PY{n}{find\PYZus{}cliques}\PY{p}{(}\PY{n}{G}\PY{p}{)}\PY{p}{)}
         \PY{n}{cliques} \PY{o}{=} \PY{p}{[}\PY{n+nb}{len}\PY{p}{(}\PY{n}{c}\PY{p}{)} \PY{k}{for} \PY{n}{c} \PY{o+ow}{in} \PY{n+nb}{sorted}\PY{p}{(}\PY{n}{max\PYZus{}cliques}\PY{p}{,} \PY{n}{key}\PY{o}{=}\PY{n+nb}{len}\PY{p}{,} \PY{n}{reverse}\PY{o}{=}\PY{n+nb+bp}{True}\PY{p}{)}\PY{p}{]}
         \PY{n+nb}{list}\PY{p}{(}\PY{n+nb}{set}\PY{p}{(}\PY{n}{cliques}\PY{p}{)}\PY{p}{)}
\end{Verbatim}

            \begin{Verbatim}[commandchars=\\\{\}]
{\color{outcolor}Out[{\color{outcolor}10}]:} [2, 3, 4, 5, 6]
\end{Verbatim}
        
    \section{Zadanie 7.}\label{zadanie-7.}

Przeprowadź grupowanie aglomeracyjne UPGMA. Wykreśl dendrogram lub jego
istotny fragment i zaproponuj arbitralny podział grafu.

Grupowanie aglomeracyjne wymaga macierzy odległości dla wierzchołków w
grafie. Aby stworzyć taką macierz, która jako odległość przyjmuje ilość
krawędzi przez które trzeba przejść idąc z wierzchołka A do B
(najkrótsza ścieżka). Macierz została stworzona za pomocą algorytmu
Floyda:

    \begin{Verbatim}[commandchars=\\\{\}]
{\color{incolor}In [{\color{incolor}10}]:} \PY{n}{fw} \PY{o}{=} \PY{n}{nx}\PY{o}{.}\PY{n}{floyd\PYZus{}warshall\PYZus{}numpy}\PY{p}{(}\PY{n}{G}\PY{p}{)}
\end{Verbatim}

    A następnie macierz została użyta w grupowaniu:

    \begin{Verbatim}[commandchars=\\\{\}]
{\color{incolor}In [{\color{incolor}20}]:} \PY{o}{\PYZpc{}}\PY{k}{matplotlib} inline
         \PY{k+kn}{import} \PY{n+nn}{matplotlib.pyplot} \PY{k+kn}{as} \PY{n+nn}{plt}
         \PY{k+kn}{import} \PY{n+nn}{scipy.cluster}
         \PY{n}{plt}\PY{o}{.}\PY{n}{figure}\PY{p}{(}\PY{n}{figsize}\PY{o}{=}\PY{p}{(}\PY{l+m+mi}{10}\PY{p}{,} \PY{l+m+mi}{10}\PY{p}{)}\PY{p}{)}
         \PY{n}{plt}\PY{o}{.}\PY{n}{ylabel}\PY{p}{(}\PY{l+s}{\PYZsq{}}\PY{l+s}{Distance}\PY{l+s}{\PYZsq{}}\PY{p}{)}
         \PY{n}{plt}\PY{o}{.}\PY{n}{xlabel}\PY{p}{(}\PY{l+s}{\PYZsq{}}\PY{l+s}{Nodes}\PY{l+s}{\PYZsq{}}\PY{p}{)}
         \PY{n}{z} \PY{o}{=} \PY{n}{scipy}\PY{o}{.}\PY{n}{cluster}\PY{o}{.}\PY{n}{hierarchy}\PY{o}{.}\PY{n}{linkage}\PY{p}{(}\PY{n}{fw}\PY{p}{,} \PY{n}{method}\PY{o}{=}\PY{l+s}{\PYZsq{}}\PY{l+s}{average}\PY{l+s}{\PYZsq{}}\PY{p}{)}
         \PY{n}{d} \PY{o}{=} \PY{n}{scipy}\PY{o}{.}\PY{n}{cluster}\PY{o}{.}\PY{n}{hierarchy}\PY{o}{.}\PY{n}{dendrogram}\PY{p}{(}\PY{n}{z}\PY{p}{,} \PY{n}{leaf\PYZus{}rotation}\PY{o}{=}\PY{l+m+mf}{90.}\PY{p}{,} \PY{n}{leaf\PYZus{}font\PYZus{}size}\PY{o}{=}\PY{l+m+mf}{12.}\PY{p}{)}
         \PY{n}{plt}\PY{o}{.}\PY{n}{show}\PY{p}{(}\PY{p}{)}
\end{Verbatim}

    \begin{center}
    \adjustimage{max size={0.9\linewidth}{0.9\paperheight}}{not_files/not_26_0.png}
    \end{center}
    { \hspace*{\fill} \\}
    
    Arbitralny podział grafu:


Graf nie daje się łatwo podzielić na równe części. Ze względu na
odległość wierzchołków od siebie, zaproponowałabym podzielenie grafu na
5 części: 
\begin{itemize}
	\item wierzchołek nr 14 (I grupa) 
	\item wierzchołek nr 19 (II grupa) 
	\item wierzchołek nr 12 (III grupa) 
	\item wierzchołek nr 26 (IV grupa) 
	\item pozostałe wierzchołki (V grupa)
\end{itemize}
Wynika to z faktu, że wierzchołki 12, 14, 19 i 26 są zdecydowanie dalej
od pozostałych wierzchołków i jednocześnie od siebie nawzajem.

    % Add a bibliography block to the postdoc
    
    
    
    \end{document}
